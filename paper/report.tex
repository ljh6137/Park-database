%保存为UTF-8编码格式
%用xelatex编译
 
\documentclass[UTF8,a4paper,12pt]{ctexart}
\usepackage[left=2.50cm, right=2.50cm, top=2.50cm, bottom=2.50cm]{geometry} %页边距
\CTEXsetup[format={\Large\bfseries}]{section} %设置章标题字号为Large,居左
%\CTEXsetup[number={\chinese{section}}]{section}
%\CTEXsetup[name={(,)}]{subsection}
%\CTEXsetup[number={\chinese{subsection}}]{subsection}
%\CTEXsetup[name={(,)}]{subsubsection}
%\CTEXsetup[number=\arabic{subsubsection}]{subsubsection}  %以上四行为各级标题样式设置,可根据需要做修改
 
%\linespread{1.5} %设置全文行间距
 
 
%\usepackage[english]{babel}
%\usepackage{float}     %放弃美学排版图表
\usepackage{fontspec}   %修改字体
\usepackage{amsmath, amsfonts, amssymb} % 数学公式相关宏包
\usepackage{color}      % color content
\usepackage{graphicx}   % 导入图片
\usepackage{subfigure}  % 并排子图
\usepackage{url}        % 超链接
\usepackage{bm}         % 加粗部分公式,比如\bm{aaa}aaa
\usepackage{multirow}
\usepackage{booktabs}
\usepackage{epstopdf}
\usepackage{epsfig}
\usepackage{longtable}  %长表格
\usepackage{supertabular}%跨页表格
\usepackage{algorithm}
\usepackage{algorithmic}
\usepackage{changepage}
\usepackage{listings}
\usepackage{xcolor}

 
 
%%%%%%%%%%%%%%%%%%%%%%%
% -- text font --
% compile using Xelatex
%%%%%%%%%%%%%%%%%%%%%%%
% -- 中文字体 --
%\setCJKmainfont{Microsoft YaHei}  % 微软雅黑
%\setCJKmainfont{YouYuan}  % 幼圆
%\setCJKmainfont{NSimSun}  % 新宋体
%\setCJKmainfont{KaiTi}    % 楷体
\setCJKmainfont[AutoFakeBold=true]{SimSun}   % 宋体
%\setCJKmainfont{SimHei}   % 黑体
 
% -- 英文字体 --
\setmainfont{Times New Roman}
%\setmainfont{DejaVu Sans}
%\setmainfont{Latin Modern Mono}
%\setmainfont{Consolas}
%
%
\renewcommand{\algorithmicrequire}{ \textbf{Input:}}     % use Input in the format of Algorithm
\renewcommand{\algorithmicensure}{ \textbf{Initialize:}} % use Initialize in the format of Algorithm
\renewcommand{\algorithmicreturn}{ \textbf{Output:}}     % use Output in the format of Algorithm
\renewcommand{\abstractname}{\textbf{\large {摘\quad 要}}} %更改摘要二字的样式
\newcommand{\xiaosi}{\fontsize{12pt}{\baselineskip}}     %\xiaosi代替设置12pt字号命令,不加\selectfont,行间距设置无效
\newcommand{\wuhao}{\fontsize{10.5pt}{10.5pt}\selectfont}
 
\usepackage{fancyhdr} %设置全文页眉、页脚的格式
\pagestyle{fancy}
\lhead{}           %页眉左边设为空
\chead{}           %页眉中间
\rhead{}           %页眉右边
%\rhead{\includegraphics[width=1.2cm]{1.eps}}  %页眉右侧放置logo
\lfoot{}          %页脚左边
\cfoot{\thepage}  %页脚中间
\rfoot{}          %页脚右边
 
 
%%%%%%%%%%%%%%%%%%%%%%%
%  设置水印
%%%%%%%%%%%%%%%%%%%%%%%
%\usepackage{draftwatermark}         % 所有页加水印
%\usepackage[firstpage]{draftwatermark} % 只有第一页加水印
% \SetWatermarkText{Water-Mark}           % 设置水印内容
% \SetWatermarkText{\includegraphics{fig/ZJDX-WaterMark.eps}}         % 设置水印logo
% \SetWatermarkLightness{0.9}             % 设置水印透明度 0-1
% \SetWatermarkScale{1}                   % 设置水印大小 0-1
 
\usepackage{hyperref} %bookmarks
\hypersetup{colorlinks, bookmarks, unicode} %unicode
 
\lstset{
  basicstyle=\ttfamily\normalsize,
  keywordstyle=\mathcal,
  numbers=left,
  numberstyle=\tiny,
  frame=single,
  breaklines=true,
  backgroundcolor=\color{gray!10},
  keywordstyle=\color{blue}\bfseries,
  commentstyle=\color{orange!50!black},
  stringstyle=\color{red!50!black},
} 
 
\title{\textbf{\Large{车辆租赁管理系统实验报告}}}
\author{ 林杰泓 22336137\\刘艺凡 22336162}
%\date{\today}
%\date{2021/10/21}
 
 
 
\begin{document}
 
\maketitle
%\tableofcontents
 
% \begin{abstract}
% 本模板可以提供一般性单栏文档的生成,可以根据需要选择是否要目录、摘要,可自行选择日期生成方式,参考文献使用交叉引用条目形式使用,便于编辑和管理。务必注意,latex编译器需要选择xelatex.
% \end{abstract}
 
% \begin{center}
% \large{\textbf{Abstract}}
% \end{center}
 
% \begin{adjustwidth}{1cm}{1cm}
% \hspace{1.5em}Here is the first par. of abstract.Here is the first par. of abstract.
 
% \noindent\hspace{1.5em}Here is the second par. of abstract.Here is the second par. of abstract.
% \end{adjustwidth}
 
%\thispagestyle{empty}       %本页不显示页码
%\newpage                    %分页
%%\tableofcontents\thispagestyle{empty}
%\newpage
%\setcounter{page}{1}        %从下面开始编页,页脚格式为导言部分设置的格式
\newpage 
\tableofcontents
\newpage
\section{实验题目}
设计一个车辆租赁管理系统,包括车辆信息管理、租赁管理、客户管理等功能。车辆信息管理负责车辆
信息的添加、修改和查询;租赁管理负责租赁信息的录入、修改和查询;客户管理负责客户信息的添
加、修改和查询。
\section{需求分析}

\subsection{功能需求}

\begin{itemize}
    \item 车辆信息管理:包括车辆的添加、修改、查询。每辆车有编号、品牌、型号、车牌号、租金等信息。
    \item 租赁管理:包括租赁记录的管理,租赁客户、租赁车辆等。
    \item 客户管理:管理客户信息,包含客户ID、姓名、联系方式等。
\end{itemize}

\subsection{非功能需求}

\begin{itemize}
    \item 安全性:对用户的权限进行控制,确保只有管理员可以进行修改操作。
    \item 系统响应时间:保证系统能快速响应用户操作,尤其是查询操作。
    \item 界面友好性:设计直观的用户界面,保证用户能够方便地操作和管理信息。
\end{itemize}

\section{系统设计}

\subsection{系统结构图}

\begin{figure}[htbp]  % figure 环境用于插入图片并进行浮动
    \centering  % 图片居中
    \includegraphics[width=1\textwidth]{pic/sap.png}  % 设置图片宽度为文档宽度的一半
    \caption{系统结构图}  % 图片标题
    \label{fig:sap}  % 图片标签,便于引用
\end{figure}

车辆租赁管理系统采用分层架构设计,
主要分为表现层、业务逻辑层和数据访问层,各层次的功能和作用如下:

\begin{itemize}
    \item 表现层(前端):\\
    表现层是系统与用户交互的部分,负责用户界面的显示和操作处理,
    包括车辆信息的查询、客户信息录入、租赁信息的修改等功能。
    具体技术采用HTML、CSS等前端技术,确保界面美观和用户体验友好。
    \item 业务逻辑层(后端):\\
    业务逻辑层位于表现层与数据访问层之间,
    是系统功能实现的核心部分,主要负责接收前端请求、
    处理业务逻辑并与数据库交互。车辆管理、租赁管理和客户管理的功能均在该层实现。
    本系统采用Django框架开发业务逻辑层,
    支持高效的请求响应和模块化开发。
    
    \item 数据访问层(数据库):\\
    数据访问层是系统的数据存储和管理部分,负责与数据库交互,
    执行SQL查询操作以实现数据的增删改查。
    本系统选用PostgreSQL作为数据库管理系统,
    设计了满足第三范式(3NF)的数据库结构,确保数据存储的规范性和完整性。
\end{itemize}

各层通过接口进行数据交互:表现层将用户请求发送给业务逻辑层,
业务逻辑层根据功能需求调用数据访问层与数据库交互,
最终返回结果至表现层供用户查看和操作。
这种分层设计提升了系统的可维护性和扩展性。

针对需求分析,我们得到了系统结构图,如图\ref{fig:sap}所示。

\subsection{系统功能模块图}

\begin{figure}[htbp]  % figure 环境用于插入图片并进行浮动
    \centering  % 图片居中
    \includegraphics[width=1\textwidth]{pic/system_function_modules.png}  % 设置图片宽度为文档宽度的一半
    \caption{系统功能模块图}  % 图片标题
    \label{fig:sfm}  % 图片标签,便于引用
\end{figure}

车辆租赁管理系统的功能模块图以系统需求为基础,分为三个主要功能模块:
车辆信息管理模块、
租赁管理模块和客户信息管理模块。各模块的功能和相互关系描述如下:

% \vspace{-0.3cm}
\begin{itemize}
    \item 车辆信息管理模块:实现车辆信息的添加、修改等功能。\\
    添加功能用于录入车辆基本信息,如车辆编号、车型、租赁价格等。\\
    修改功能用于更新车辆状态或属性,如车辆的租赁状态、等。
    \item 租赁管理模块:实现租赁交易的管理,包括租赁信息的录入、修改和查询。\\
    录入功能记录租赁交易信息,如租赁车辆、客户、起始日期、结束日期及费用等。\\
    修改功能允许调整租赁的相关信息,如租赁时间或费用的变更。\\
    查询功能支持客户查找租赁记录,方便查看历史交易。
    \item 客户信息管理模块:实现客户信息的管理,包括添加、修改和查询功能。\\
    添加功能用于录入新客户的信息,如姓名、联系方式、身份证号等。\\
    修改功能用于更新客户信息,如更改联系方式或地址。\\
    查询功能支持按客户姓名或其他条件检索客户信息,便于用户管理客户数据。\\
\end{itemize}
\vspace{-0.8cm}
针对需求分析,我们得到了系统的功能模块图,如图\ref{fig:sfm}所示。

\subsection{E-R图}

\begin{figure}[htbp]  % figure 环境用于插入图片并进行浮动
    \centering  % 图片居中
    \includegraphics[width=1\textwidth]{pic/er.png}  % 设置图片宽度为文档宽度的一半
    \caption{E-R图}  % 图片标题
    \label{fig:er}  % 图片标签,便于引用
\end{figure}

车辆租赁管理系统的E-R图描述了系统中的主要实体及其之间的关系,
反映了系统的数据结构和逻辑关系。根据系统需求分析,设计了以下实体及其属性:

\subsubsection{实体及属性}

\begin{itemize}
    \item 客户(Customer)属性:\\
    用户(User):与Django的内置用户模型进行一对一关联,扩展用户信息。\\
    ID(主键):客户唯一标识符。\\
    姓名(name):客户姓名。\\
    联系方式(contact):客户的联系方式。\\
\vspace{-0.7cm}
    \item 车辆(Vehicle)属性:\\
    型号(Model,主键):车辆的唯一标识。\\
    价格(Price):车辆的租赁价格。\\

    \item 车辆库(Repository)属性:\\
    车辆ID(Car\_ID,主键):标识库中每辆车的唯一编号。\\
    是否租赁(Is\_leased):表示车辆是否已被租赁。\\
\vspace{-0.7cm}
    \item 车辆信息(Info)属性:\\
    车辆ID(Car\_ID,外键):引用车辆库中的车辆ID。\\
    型号(Model,外键):引用车辆表中的型号信息。\\
\vspace{-0.7cm}    
    \item 租赁记录(Lease)属性:\\
    车辆ID(Car\_ID,外键):关联车辆库中的车辆。\\
    客户ID(ID,外键):关联客户表中的客户。\\
\end{itemize}

\subsubsection{实体之间的关系} 
\begin{itemize}
\item 客户与租赁记录:
一个客户可以进行多次租赁记录(一对多)。

\item 车辆与车辆库:
车辆库中的每辆车属于一个车辆型号(多对一)。

\item 车辆库与租赁记录:
每辆车在租赁记录中最多出现一次(一对一)。

\item 车辆库与车辆信息:
每辆车在车辆信息表中有对应的详细记录(一对一)。
\end{itemize}

针对需求分析,画出E-R图表示的概念模型,如图\ref{fig:er}所示。

\subsection{数据库模式}




根据车辆租赁管理系统的需求和 E-R 图,将概念模型转换为关系模型,设计出满足第三范式(3NF)的数据库模式。具体如下:

\subsubsection{表设计}


\begin{table}[h!]
    \centering
    \caption{客户表(Customer)}
\begin{tabular}{|l|l|l|l|}
\hline
字段名 & 数据类型 & 约束 & 说明 \\
\hline
ID & VARCHAR(10) & PRIMARY KEY & 客户唯一标识符 \\
\hline
user\_id & INTEGER & UNIQUE, NOT NULL & 关联 Django 用户表 \\
\hline
name & VARCHAR(50) & NOT NULL & 客户姓名 \\
\hline
contact & VARCHAR(15) & NOT NULL & 客户联系方式 \\
\hline
\end{tabular}
\end{table}


\begin{table}[h!]
    \centering
    \caption{车辆表(Vehicle)}
\begin{tabular}{|l|l|l|l|}
\hline
字段名 & 数据类型 & 约束 & 说明 \\
\hline
Model & VARCHAR(50) & PRIMARY KEY & 车辆型号 \\
\hline
Price & INTEGER & NOT NULL & 租赁价格 \\
\hline
\end{tabular}
\end{table}


\begin{table}[h!]
    \centering
    \caption{车辆库表(Repository)}
\begin{tabular}{|l|l|l|l|}
\hline
字段名 & 数据类型 & 约束 & 说明 \\
\hline
Car\_ID & VARCHAR(10) & PRIMARY KEY & 车辆唯一标识符 \\
\hline
Is\_leased & BOOLEAN & DEFAULT FALSE & 是否被租赁 \\
\hline
\end{tabular}
\end{table}


\begin{table}[h!]
    \centering
    \caption{车辆信息表(Info)}
\begin{tabular}{|l|l|l|l|}
\hline
字段名 & 数据类型 & 约束 & 说明 \\
\hline
Car\_ID & VARCHAR(10) & FOREIGN KEY (Repository.Car\_ID), NOT NULL & 关联车辆库表 \\
\hline
Model & VARCHAR(50) & FOREIGN KEY (Vehicle.Model), NOT NULL & 关联车辆型号 \\
\hline
\end{tabular}
\end{table}


\begin{table}[h!]
    \centering
    \caption{租赁记录表(Lease)}
\begin{tabular}{|l|l|l|l|}
\hline
字段名 & 数据类型 & 约束 & 说明 \\
\hline
ID & VARCHAR(10) & FOREIGN KEY (Customer.ID), NOT NULL & 关联客户表 \\
\hline
Car\_ID & VARCHAR(10) & FOREIGN KEY (Repository.Car\_ID), NOT NULL & 关联车辆库表 \\
\hline
\end{tabular}
\end{table}

\subsubsection{数据库模式特点}

\begin{enumerate}
    \item \textbf{满足第三范式(3NF)}
    \begin{itemize}
        \item 消除冗余:每个表只存储一种实体的属性,避免数据冗余。
        \item 数据依赖明确:所有非主键字段完全依赖于主键。
    \end{itemize}
    \item \textbf{完整性约束}
    \begin{itemize}
        \item 主键约束:每个表都有明确的主键,保证数据唯一性。
        \item 外键约束:通过外键建立表间关系,保证数据一致性。
    \end{itemize}
    \item \textbf{扩展性}
    \begin{itemize}
        \item 模块化表设计便于功能扩展,如增加车辆维修管理、客户等级管理等功能。
    \end{itemize}
\end{enumerate}


\subsection{安全性与完备性}


为了确保车辆租赁管理系统的数据库安全性与完备性,采取了以下设计和实现措施:

\subsubsection{用户认证与权限管理}
\begin{itemize}
    \item \textbf{用户认证}:通过 Django 内置的用户认证系统(Authentication System),对系统用户进行登录验证,确保只有授权用户能够访问系统。
    \item \textbf{权限控制}:基于角色分配权限,不同用户(如管理员和普通用户)拥有不同的数据库访问权限。例如:
    \begin{itemize}
        \item 管理员:拥有添加、修改、删除数据的权限。
        \item 普通用户:仅能查看车辆信息和提交租赁请求。
    \end{itemize}
\end{itemize}

\subsubsection{数据完整性约束}
\begin{itemize}
    \item \textbf{主键约束}:每张表都有主键,确保每条记录具有唯一标识符。
    \item \textbf{外键约束}:外键关系保证了数据的一致性,例如租赁记录表中的车辆 ID 和客户 ID 必须合法且存在。
    \item \textbf{非空约束}:关键字段(如客户姓名、联系方式等)设置为非空,确保数据完整。
    \item \textbf{默认值约束}:为布尔字段(如车辆是否租赁)设置默认值,减少人为错误。
\end{itemize}

\subsubsection{数据保护与加密}
\begin{itemize}
    \item \textbf{敏感数据加密}:对于用户的敏感信息(如密码),使用 Django 提供的密码哈希功能进行加密存储。
\end{itemize}

Django 的 User 模型已经内置了密码哈希功能,
因此只需要使用 create\_user 或 set\_password 方法来处理用户密码。
下面是我们实现的思路:

\begin{lstlisting}[language=python]
    # 创建用户并加密密码
    user = User.objects.create_user(username=username, password=password)
\end{lstlisting}

\subsubsection{日志记录与审计}
\begin{itemize}
    \item 系统记录所有数据库操作的日志,包括数据添加、修改、删除等关键操作。
    \item 日志可用于追踪用户操作行为,及时发现和响应潜在的安全威胁。
\end{itemize}

\begin{lstlisting}[language=python]
    LOGGING = {
    'version': 1,
    'disable_existing_loggers': False,
    'formatters': {
        'verbose': {
            'format': '{levelname} {asctime} {module} {message}',
            'style': '{',
        },
        'simple': {
            'format': '{levelname} {message}',
            'style': '{',
        },
    },
    'handlers': {
        'console': {
            'level': 'DEBUG',
            'class': 'logging.StreamHandler',
            'formatter': 'simple',
        },
        'file': {
            'level': 'INFO',
            'class': 'logging.FileHandler',
            'filename': 'myapp.log',
            'formatter': 'verbose',
        },
    },
    'loggers': {
        'django': {
            'handlers': ['console'],
            'level': 'DEBUG',
            'propagate': True,
        },
        'myapp': {
            'handlers': ['console', 'file'],
            'level': 'DEBUG',
            'propagate': True,
        },
    },
}
\end{lstlisting}

在views.py中,我们可以记录用户操作日志,如下所示:

\begin{lstlisting}[language=python]
    if User.objects.filter(username=username).exists():
            logger.warning(f"Username {username} already exists.")
            messages.error(request, 'Username already exists. Please choose another one.')
            return render(request, 'management/register.html')
\end{lstlisting}

然后我们操作后就可以在myapp.log文件中看到相关日志。

\begin{lstlisting}[language=bash]
    WARNING 2024-12-25 10:36:47,223 views Username demo already exists.
    INFO 2024-12-25 10:41:41,707 views User demo logged in successfully.
    INFO 2024-12-25 10:41:43,616 views User demo returned vehicle with ID XPENG001.
    INFO 2024-12-25 10:41:54,603 views User demo returned vehicle with ID XPENG001.
\end{lstlisting}

\subsubsection{最小权限原则}
\begin{itemize}
    \item 数据库账户的权限设置遵循最小权限原则,只赋予完成任务所需的最低权限,避免权限过高导致的安全隐患。
\end{itemize}

通过以上安全性设计,系统能够有效保护数据免受未授权访问、篡改或丢失,确保数据库的完整性、保密性和可用性。

\section{最终效果展示}



\end{document}

